% LaTeX file for resume
% This file uses the resume document class (res.cls)
\documentclass{res}
%\usepackage{helvetica} % uses helvetica postscript font (download helvetica.sty)
%\usepackage{newcent}   % uses new century schoolbook postscript font
\newsectionwidth{0pt}  % So the text is not indented under section headings
\usepackage{fancyhdr}  % use this package to get a 2 line header
%\usepackage{hyperref}
\renewcommand{\headrulewidth}{0pt} % suppress line drawn by default by fancyhdr
\setlength{\headheight}{24pt} % allow room for 2-line header
\setlength{\headsep}{24pt}  % space between header and text
\setlength{\headheight}{24pt} % allow room for 2-line header
\pagestyle{fancy}     % set pagestyle for document
%\rhead{ {\it Linjun Yang}\\{\it p. \thepage} } % put text in header (right side)
\cfoot{}                                     % the foot is empty
\topmargin=-0.5in % start text higher on the page

\begin{document}
\thispagestyle{empty} % this page has no header
\name{\Large{Linjun Yang} \\[12pt]}

\address{}

%\address{23516 NE 22nd st, \\Sammamish, WA, 98074, USA}

\address{Phone:  +1-425-505-9910 \\
Email: linjun.yang@hotmail.com
 }




\begin{resume}

\section{{}}
\emph{Dr. Linjun Yang has over 10 years experiences on researching/developing cutting-edge image retrieval/computer vision/machine learning technologies. He has authored/co-authored more than 60 peer-reviewed papers published in various conference proceedings and journals and 23 filed US/international patents. His papers have gotten over 2000 citations and his h-index is 22 according to Google Scholar 4/2015. He is now leading a team to develop various advanced image understanding technologies/systems on billions of Web image data, including deep learning, semantic embedding, duplicate-detection, visual similarity search, annotation, etc., with application to various Microsoft products.}

%\vspace{0.1in}
\section{{EDUCATION}}
%\vspace{8pt}
{\sl Ph.D}, Multimedia Information Retrieval \\
Delft University of Technology, The Netherlands \hfill 2013

{\sl M.S.}, Compute Science \\
Fudan University, Shanghai, China \hfill 2006


{\sl B.S.}, Electronics \& Information System \\ % \sl will be bold italic in
                     % New Century Schoolbook (or
                     % any postscript font) and
                     % just slanted in Computer
                     % Modern (default) font
East China Normal University, Shanghai, China      \hfill   2001

%Hanshan Middle School, Anhui, Chi na      \hfill   1997

\section{AWARDS}

The Best Paper Award at ACM International Conference on Multimedia (ACM MM), Beijing, China, 2009.
%\vspace{-8mm}

The Best Student Paper Award at the 18th ACM Conference on Information and Knowledge Management (CIKM), Hong Kong, 2009.

%Microsoft Hi-Po Program, 2011.



%\vspace{0.1in}
\section{{PROFESSIONAL EXPERIENCE}}
%\vspace{8pt}
{\sl Microsoft Bing} \hfill        Jul. 2012 - Present \\
Principal Software Engineer Manager \\
\emph{ - lead a team of 10+ people to research and develop cutting-edge computer vision and machine learning technologies, as well as develop products such as Bing visual search.}
    \begin{itemize} \itemsep 1pt % reduce space between items
   	\item Visual search (a.k.a search by image) \\
   	\emph{own the product e2e including building measurement set/metric; developing models/rankers for search quality, algorithms/pipeline for fast search, and new UX features (like object detection)for user experience; tracking/improving user engagement metric}
   	
    \item Deep learning for image understanding \\
    \emph{research/develop deep neural network models for improving various image understanding tasks including feature learning, image categorization, object detection, semantic embedding, etc.}
    
    \item Large-scale duplicate/similarity search system \\
    \emph{build a system to retrieve similar/duplicate images from billion-scale database for billions of images per day}
    \item Image tagging / recognition \\
    \emph{build algorithms/system to generate tags for any given image based on deep learning and similarity search}
    \item {Image processing pipeline} \\
    \emph{designed and implemented a graph based image processing pipeline to faciliate algorithm/model development and deployment}
    \end{itemize}

{\sl Microsoft Bing} \hfill        Nov. 2011 - Apr. 2012 \\
Visiting  Researcher, Multimedia Search Team
    \begin{itemize} \itemsep 1pt % reduce space between items
    \item Designing and implementing back-end image processing library/pipeline.
    \item Developing billion-scale image duplicate detection system.
    \end{itemize}

{\sl Microsoft Research Asia} \hfill        Jul. 2006 - Jul. 2012 \\
Associate Researcher, Media Computing Group

   \begin{itemize} \itemsep 1pt % reduce space between items
   \item Visual search

   \emph{We study various methods to improve the precision of image and video search by example.}


   \item Content-based Image Ranking/Reranking

   \emph{This objective of this project is to improve image search ranking by utilizing visual content analysis.}

   \item Video analysis SDK

    \emph{to develop a suite of libraries for image and video content
    analysis based on Windows DirectShow API. }

 \end{itemize}

%{\sl ALi (Shanghai) Corporation} \\[2pt]
%Software Engineer \hfill    Jul. 2001 -- Sep. 2002
%
%  \begin{itemize}
%  \item  \emph{to develop the device drivers for Windows and Linux.}
% \end{itemize}

\section{{COMPUTING SKILLS }}
%\vspace{0.2in}
\begin{itemize}
    \item Experienced in C++, C\#.
    \item Experienced in Machine learning and Information retrieval.
\end{itemize}


%\vspace{0.1in}
%\section{{PROFESSIONAL ACTIVITIES}}
%\vspace{0.2in}
%\begin{itemize}
%    \item Technical Program Committee: ACM MM 2013-2014; ICMR 2012; ICME 2012; ICDE 2011; ICMR 2011; ICIP 2011; SIGIR 2011-2014; IJCAI 2011; WIAMIS 2011; CIVR 2010; ICME 2009; ICME 2008.
%\end{itemize}

%\vspace{0.1in}
\section{{REPRESENTATIVE PUBLICATIONS}}
%\vspace{0.1in}
%2010
(full publicationb list: https://scholar.google.com/citations?user=cvgKxDQAAAAJ)

[10] Linjun Yang, Alan Hanjalic, ``Learning to Rerank Web Images,'' IEEE Multimedia 2013.

[9] Bo Geng$^*$, Linjun Yang, Chao Xu, Xian-Sheng Hua, Shipeng Li, ``The Role of Attractiveness in Web Image Search,'' {\it ACM Multimedia 2011}.

[8] Xinmei Tian$^*$, Linjun Yang, Jingdong Wang, Xiuqing Wu, Xian-Sheng Hua, ``Bayesian Visual Reranking,'' IEEE Transactions on Multimedia.

[7] Lifeng Shang, Linjun Yang, Fei Wang, Kwok-Ping Chan, Xian-Sheng Hua, ``Real-time large scale near-duplicate web video retrieval,'' {\it ACM Multimedia 2010.}

[6] Linjun Yang, Alan Hanjalic, ``Supervised Reranking for Web Image Search,'' {\it ACM Multimedia 2010.}

[5] Bo Geng$^*$, Linjun Yang, Chao Xu, Xian-Sheng Hua, ``Ranking Model Adaptation for Domain Specific Search,'' {\it ACM Conference on Information and Knowledge Management (CIKM) 2009 (Best Student Paper Award).}

%2009
[4] Zheng-Jun Zha$^*$, Linjun Yang, Tao Mei, Meng Wang, Zengfu Wang, ``Visual Query Suggestion," {\it ACM Multimedia 2009 (Best Paper Award).}


[3] Lei Wu$^*$, Linjun Yang, Nenghai Yu, Xian-Sheng Hua, ``Learning to
Tag'', {\it WWW '09: Proceeding of the 18th international conference
on World Wide Web. 2009.}

[2] Dong Liu$^*$, Xian-Sheng Hua, Linjun Yang, Meng Wang, Hong-Jiang
Zhang, ``Tag Ranking'', {\it WWW '09: Proceeding of the 18th
international conference on World Wide Web. 2009.}


%2008
[1] Xinmei Tian$^*$, Linjun Yang, Jingdong Wang, Yichen Yang$^*$,
Xiuqing Wu, Xian-Sheng Hua, ``Bayesian Video Search Reranking,''
{\it ACM Multimedia 2008.}

\vspace{-0.1in} \emph{* the interns I mentored at Microsoft Research
Asia}




\end{resume}
\end{document}
